\paragraph{Pregunta:} Si las bases de datos orientadas a objetos surgieron para combatir las limitaciones de las bases de datos relacionales, �por qu� siguen siendo las bases de datos relacionales las m�s utilizadas?

\paragraph{Respuesta:} Se siguen usando porque es un est�ndar muy arraigado y extendido, y en muchas empresas es muy costoso hacer un cambio a nuevas tecnolog�as aunque aporten m�s ventajas que las antiguas. Adem�s, el est�ndar de las bases de datos orientadas a objetos no est� todo lo maduro que debiera para poder hacer la migraci�n sin correr ning�n riesgo.
Por otra parte, las bases de datos orientadas a objetos suelen utilizarse en aplicaciones de dise�o y fabricaci�n por computador (CAD), que no est�n tan extendidas como otras aplicaciones comerciales que hacen uso de bases de datos relacionales. Por esta raz�n, se siguen utilizando en mayor medida, por estar m�s extendidas en las aplicaciones comerciales. 
