\section{Preguntas seleccionadas}

\paragraph{Pregunta} Como se ha comentado en el trabajo, las bases de datos deductivas se utilizan para almacenar datos de la vida real y a partir de ah� deducir informaci�n que en principio no est� almacenada en la base de datos, pero �qu� aplicaci�n real podr�a tener el uso de estas bases de datos en la vida real?

\paragraph{Respuesta} Las bases de datos deductivas est�n recomendadas cuando se tienen grandes cantidades de informaci�n y se necesitan sistemas que analicen esta informaci�n e infiera informaci�n nueva, como por ejemplo las �reas que cubre la inteligencia artificial. Algunas de estas son: 
Decission support system. En esta �rea las bases de datos deductivas ayudan a tomar decisiones para planificar el futuro relacionando la informaci�n que ya tenemos almacenada en la base de datos. 
Expert Systems. Estos sistemas son utilizados en medicina en los que se obtiene mucha informaci�n cuando �sta es monitorizada. La informaci�n se almacena en bases de datos deductivas y posteriormente se obtienen conclusiones. 
An�lisis del Genoma Humano. Sakamoto, Goto y Takagi trabajan con un sistema de bases de datos deductiva que permite hacer  miner�a de datos mediante la especificaci�n de reglas, muchas de ellas de forma recursiva, utilizando un lenguaje declarativo y no basado en procedimientos como se hace en la programaci�n tradicional. 

\clearpage

\paragraph{Pregunta} En el trabajo se habla sobre distintos modelos de bases de datos relacionales difusas, pero no se hace referencia a modelos de bases de datos entidad-interrelaci�n. Nombre alg�n ejemplo de modelo de base de datos entidad-interrelaci�n difusa.

\paragraph{Respuesta} Un modelo entidad-interrelaci�n difuso es el llamado FuzzyEER. Es una extensi�n del modelo entidad-interrelaci�n que incluye un conjunto de constructores para manejar informaci�n imprecisa.

En FuzzyERR se maneja el concepto de grado para indicar el nivel en que un elemento del esquema cumple con una determinada caracter�stica.
Dichos grados pueden ser:

\begin{milista}

\item \textbf{Grado de pertenencia:} La pertenencia de un valor a una instancia concreta puede ser cuantificada por un grado.

\item \textbf{Grado de satisfacci�n:} En una instancia, una propiedad puede cumplirse con cierto grado entre dos extremos.

\item \textbf{Grado de incertidumbre:} Expresa la certeza con la que conocemos un dato determinado para una instancia concreta.

\item \textbf{Grado de posibilidad:} Mide la posibilidad de la informaci�n  que se est� modelando para cada instancia de la entidad.

\item \textbf{Grado de importancia:} Distintos valores de un atributo pueden tener diferentes importancias de forma que pueden existir distintos valores para un mismo atributo que tengan distinta importancia.

\end{milista}

\clearpage

\section{Pregunta descartada}

\paragraph{Pregunta}

En el trabajo se habla sobre el lenguaje FSQL, sus caracter�sticas, comparadores y umbral de cumplimiento, pero no se muestra ning�n ejemplo de uso 
de la sintaxis. Muestre alg�n ejemplo.

\paragraph{Respuesta}

Suponiendo que tenemos una base de datos con informaci�n sobre personas:

''Recuperar todas las personas cuya edad es aproximadamente 12 a�os:''

\begin{verbatim}
	SELECT * FROM Personas
	WHERE Edad FEQ #20 THOLD 0.6;
\end{verbatim}

''Recuperar todas las personas m�s o menos Morenas (con grado m�nimo 0.3) cuya edad es posiblemente superior a Joven (con grado m�nimo 0.6):''

\begin{verbatim}
	SELECT * FROM Personas WHERE
	Pelo FEQ $Rubio THOLD 0.5 AND
	Edad FGT $Joven THOLD 0.8;
\end{verbatim}

