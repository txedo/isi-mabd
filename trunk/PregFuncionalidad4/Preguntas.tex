% Tipo de documento. En este caso es un art�culo, para folios A4, tama�o de la fuente 11pt y con p�gina separada para el t�tulo
\documentclass[a4paper,12pt,titlepage]{article}

% Carga de paquetes necesarios. OrdenesArticle es un paquete personalizado
\usepackage[spanish]{babel} 
\RequirePackage[T1]{fontenc}
\RequirePackage[ansinew]{inputenx} 
\usepackage[spanish,cap,cont,title,fancy]{OrdenesArticle}
\usepackage{array}
\usepackage{graphicx}
\usepackage{hyperref}
\usepackage{pifont}
\usepackage{listings}
\usepackage[usenames,dvipsnames]{color}
\usepackage{colortbl}
\usepackage{makeidx}
\hypersetup{bookmarksopen,bookmarksopenlevel=3,linktocpage,colorlinks,urlcolor=blue,citecolor=blue,
						linkcolor=blue,filecolor=blue,pdfnewwindow,
						pdftitle={Listado de preguntas Funcionalidad 4}, 
						pdfauthor={Juan Andrada Romero, Juan Jos� Antequera Flores},
						pdfsubject={Modelos Avanzados de Base de Datos}}


% Macro para definir una lista personalizada 
\newenvironment{milista}%
{\begin{list}{\textbullet}%
{\settowidth{\labelwidth}{\textbullet} \setlength{\leftmargin}{\dimexpr\labelsep+\labelwidth+5pt}
\setlength{\itemsep}{\dimexpr 0.5ex plus 0.25ex minus 0.25ex}
\setlength{\parsep}{\itemsep}
\setlength{\partopsep}{\itemsep}
\addtolength{\topsep}{-7.5pt}
}}%
{\end{list}}


\begin{document}

% En las p�ginas de portada e �ndices, no hay encabezado ni pie de p�gina
\pagestyle{empty} 

% Se incluye la portada
\thispagestyle{empty}
\begin{center}
  {\LARGE UNIVERSIDAD DE CASTILLA-LA MANCHA} \\
  \bigskip
  {\Large ESCUELA SUPERIOR DE INFORM�TICA} \\
  \vspace{28mm}
  \includegraphics[scale=0.45, keepaspectratio]{./imagenes/esi_bw.png} \\
  \vspace{30mm}
  {\LARGE \textbf{Modelos Avanzados de Base de Datos}} \\
  \vspace{14mm}
  {\Large \textsf{\textsc{Listado de preguntas del trabajo Distribuci�n 1:}}} \\
  \vspace{2mm}
  {\Large \textsf{\textsc{Bases de Datos Distribuidas}}}\\
  \vspace{15mm}
  {\large Juan Andrada Romero} \\
  {\large Juan Jos� Antequera Flores} \\
  {\large Jose Domingo L�pez L�pez} \\
  \vspace{9mm}
  {\large \today}
\end{center}
\clearpage

\parskip = 10pt
% Comienza el contenido del documento. Se utilizan n�meros ar�bigos y el encabezado y pie de p�gina personalizado
\pagenumbering{arabic}
\pagestyle{fancy}

\section{Preguntas seleccionadas}

\paragraph{Pregunta} En el trabajo se comentan las bases de datos nativas para XML. �Estas bases de datos soportan restricciones de integridad referencial, al igual que una base de datos relacional?

\paragraph{Respuesta} Seg�n \cite{XML}, la integridad referencial en bases de datos XML nativas se refiere a asegurar que las referencias que aparecen dentro de un documento XML apuntan a otros documentos existentes y v�lidos. Adem�s, existen dos tipos de integridad referencial en las bases de datos XML: integridad interna (punteros dentro de un mismo documento) e integridad externa (punteros entre documentos). 

La mayor�a de las bases de datos XML nativas realizan la validaci�n de las referencias internas al insertar los documentos en la base de datos. Tambi�n se validan las referencias internas cuando se hacen modificaciones a nivel de documento (se elimina y luego se vuelve a insertar el documento). \\
\indent En cuanto a la integridad externa, la mayor�a de las bases de datos XML nativas no la soportan.

%%%%%%%%%%%%%%%%%%%%%%%%%
\paragraph{Pregunta} En la p�gina 18 del trabajo, donde se trata acerca de XQuery, s�lo se hace referencia a �ste como un lenguaje de consulta. �Permite XQuery la inserci�n o modificaci�n de datos en una base de datos XML? En caso negativo, �de qu� formas se pueden manipular los datos en una base de datos XML?

\paragraph{Respuesta} S�lo se hace menci�n del lenguaje XQuery como lenguaje de consulta porque �ste carece de la posibilidad de realizar inserciones o modificaciones en los datos. No obstante, Microsoft ha desarrollado una extensi�n de dicho lenguaje para permitir este tipo de operaciones, llamada XML Data Modification Language (XML DML) \cite{xml-dml}.
% http://msdn.microsoft.com/en-us/library/ms177454.aspx


%%%%%%%%%%%%%%%%%%%%%%%%%%%
\paragraph{Pregunta} A d�a de hoy, el comercio electr�nico es una pr�ctica llevada a cabo por muchas empresas y que genera grandes beneficios. �Ser�a conveniente utilizar una base de datos XML en este tipo de sistemas?

En sistemas de comercio electr�nico en particular, as� como en sistemas que manejan grandes vol�menes de datos y en los que el tiempo de respuesta es cr�tico, las bases de datos XML no son la mejor opci�n ya que cada vez que se realiza una consulta es necesario recorrer todo el �rbol. Para este tipo de sistemas es m�s eficiente utilizar bases de datos relacionales u orientadas a objetos.


\clearpage


\end{document}

