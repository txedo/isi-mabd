\paragraph{Pregunta:} En el documento se comentan algunas caracter�sticas de los SGBDOR, pero no se mencionan las de un SGBDOO. �Cu�les son las caracter�sticas que debe un cumplir un SGBDOO para que sea considerado como tal?

\paragraph{Respuesta:} En 1989 se hizo el Manifiesto \cite{manifest} de los sistemas de base de datos orientados a objetos el cual propuso trece caracter�sticas obligatorias para un SGBDOO y cuatro opcionales, las cuales son:
\begin{itemize}
\item Caracter�sticas obligatorias de orientaci�n a objetos:
\begin{enumerate}
\item Deben soportarse objetos complejos
\item Deben soportarse mecanismos de identidad de los objetos
\item Debe soportarse la encapsulaci�n
\item Deben soportarse los tipos o clases
\item Los tipos o clases deben ser capaces de heredar de sus ancestros
\item Debe soportarse el enlace din�mico
\item El DML debe ser computacionalmente complejo
\item El conjunto de todos los tipos de datos debe ser ampliable
\end{enumerate}
\item Caracter�sticas obligatorias de SGBD:
\begin{enumerate}
\item Debe proporcionarse persistencia a los datos
\item El SGBD debe ser capaz de gestionar bases de datos de muy gran tama�o
\item El SGBD debe soportar a usuarios concurrentes
\item El SGBD debe ser capaz de recuperarse de fallos hardware y software
\item El SGBD debe proporcionar una forma simple de consultar los datos.
\end{enumerate}
\item Caracter�sticas opcionales:
\begin{enumerate}
\item Herencia m�ltiple
\item Comprobaci�n de tipos e inferencia de tipos
\item Sistema de base de datos distribuido
\item Soporte de versiones
\end{enumerate}
\end{itemize}