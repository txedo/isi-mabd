\paragraph{Pregunta:} En el trabajo se comentan los lenguajes de definici�n de objetos y de consulta de objetos que define el est�ndar ODMG, pero no se hace referencia a ning�n lenguaje para poder manipular objetos, es decir, insertarlos en la base de datos, modificarlos o eliminarlos. �Dicho est�ndar define un lenguaje de manipulaci�n de datos espec�fico? �C�mo se realizar�an las manipulaciones sobre objetos en una base de datos orientada a objetos?

\paragraph{Respuesta:} El est�ndar ODMG-93 sugiere que el lenguaje de manipulaci�n de objetos (OML) sea la extensi�n de un lenguaje de programaci�n, de forma que se pueden realizar entre otras las siguientes operaciones sobre la base de datos: creaci�n, borrado, y modificaci�n de un objeto. \cite{bdoo}.

Por tanto, los objetos se manipulan dentro del lenguaje de programaci�n orientado a objetos, utilizando el concepto de transacci�n y de clases persistentes. De este modo, como indica el ODMG API (\cite{odmgapi}), para realizar las operaciones de inserci�n, modificaci�n o eliminaci�n de un objeto dentro del lenguaje de programaci�n, se proceder�a de la siguiente forma:
\begin{enumerate}
	\item Crear una transacci�n.
	\item Obtener la base de datos en uso.
	\item Ejecutar la operaci�n de inserci�n, modificaci�n o eliminaci�n.
\end{enumerate}

Un ejemplo de inserci�n ser�a el siguiente:

\begin{verbatim}
 public static void storeNewProduct(Product product)
	    Transaction tx = odmg.newTransaction();
	    tx.begin();
	    // get current used Database instance
	    Database db = odmg.getDatabase(null);
	    // make persistent new object
	    db.makePersistent(product);
	    tx.commit();
\end{verbatim}

