\section{Preguntas seleccionadas}

\paragraph{Pregunta} En el trabajo se habla de los departamentos de una organizaci�n como productor de informaci�n, pero los productores pueden ser externos a nuestra organizaci�n. Esto representa un problema debido a la falta de heterogeneidad en la estructuraci�n de los datos en los almacenes de datos. �Es posible procesar los datos de manera autom�tica una vez obtenidos los mismos?

\paragraph{Respuesta} No es posible procesar de manera autom�tica datos porque la representaci�n de los mismos (diminutivos en los nombres, campos compuestos representados como uno solo, etc) depende fuertemente de la base de datos en la que est�n almacenados. Por tanto antes de adoptar los datos externos es necesario conocer detalladamente la estructura del almac�n del que extraeremos la informacion para seleccionar las correspondencias de los campos de varios almacenes para poder hacer una fusi�n lo m�s homog�nea posible.

\vspace{20mm}

\paragraph{Pregunta} En la p�gina 18 del trabajo, donde se trata acerca de XQuery, s�lo se hace referencia a �ste como un lenguaje de consulta. �Permite XQuery la inserci�n o modificaci�n de datos en una base de datos XML? En caso negativo, �de qu� formas se pueden manipular los datos en una base de datos XML?

\paragraph{Respuesta} S�lo se hace menci�n del lenguaje XQuery como lenguaje de consulta porque �ste carece de la posibilidad de realizar inserciones o modificaciones en los datos. No obstante, Microsoft ha desarrollado una extensi�n de dicho lenguaje para permitir este tipo de operaciones, llamada XML Data Modification Language (XML DML) \cite{xml-dml}.

\clearpage
\section{Preguntas descartadas}

\paragraph{Pregunta} En el trabajo se comentan las bases de datos nativas para XML. �Estas bases de datos soportan restricciones de integridad referencial, al igual que una base de datos relacional?

\paragraph{Respuesta} Seg�n \cite{XML}, la integridad referencial en bases de datos XML nativas se refiere a asegurar que las referencias que aparecen dentro de un documento XML apuntan a otros documentos existentes y v�lidos. Adem�s, existen dos tipos de integridad referencial en las bases de datos XML: integridad interna (punteros dentro de un mismo documento) e integridad externa (punteros entre documentos). 

La mayor�a de las bases de datos XML nativas realizan la validaci�n de las referencias internas al insertar los documentos en la base de datos. Tambi�n se validan las referencias internas cuando se hacen modificaciones a nivel de documento (se elimina y luego se vuelve a insertar el documento). \\
\indent En cuanto a la integridad externa, la mayor�a de las bases de datos XML nativas no la soportan.

%%%%%%%%%%%%%%%%%%%%%%%%%

\vspace{20mm}
%%%%%%%%%%%%%%%%%%%%%%%%%%%
\paragraph{Pregunta} A d�a de hoy, el comercio electr�nico es una pr�ctica llevada a cabo por muchas empresas y que genera grandes beneficios. �Ser�a conveniente utilizar una base de datos XML en este tipo de sistemas?

\paragraph{Respuesta} En sistemas de comercio electr�nico en particular, as� como en sistemas que manejan grandes vol�menes de datos y en los que el tiempo de respuesta es cr�tico, las bases de datos XML no son la mejor opci�n ya que cada vez que se realiza una consulta es necesario recorrer todo el �rbol. Para este tipo de sistemas es m�s eficiente utilizar bases de datos relacionales u orientadas a objetos.


\clearpage


