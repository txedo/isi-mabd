\section{Preguntas seleccionadas}

\paragraph{Pregunta} En el trabajo se comentan las bases de datos nativas para XML. �Estas bases de datos soportan restricciones de intregidad referencial, al igual que una base de datos relacional?

\paragraph{Respuesta} Seg�n \cite{XML}, la integridad referencial en bases de datos XML nativas se refiere a asegurar que loas referencias que aparecen dentro de un documento XML apuntan a otros documentos existentes y v�lidos. Adem�s, existen dos tipos de integridad referencial en las bases de datos XML: integridad interna (punteros dentro de un mismo documento) e integridad externa (punteros entre documentos). 

La mayor�a de las bases de datos XML nativas realizan la validaci�n de las referencias internas al insertar los documentos en la base de datos. Tambi�n se validan las referencias internas cuando se hacen modificaciones a nivel de documento (se elimina y luego se vuelve a insertar el documento). \\
\indent En cuanto a la integridad externa, la mayor�a de las bases de datos XML nativas no la soportan.


\clearpage

