\section{Comparativa}

\subsection{Similitudes}
\begin{milista}
	\item Ambas bases de datos deben  soportar grandes vol�menes de datos.
	\item Aunque no es estrictamente necesario que las bases de datos multimedia sean concurrentes, es aconsejable, y para las bases de datos web es obligatorio, ya que debe dar servicio a muchos usuarios al mismo tiempo.
	\item Una base de datos multimedia puede utilizarse como base de datos web siempre y cuando disponga de la capa de comunicaci�n entre el SGBD y el usuario que est� utilizando la aplicaci�n.
	\item Los dos tipos de bases de datos pueden utilizar sistemas gestores de bases de datos tanto relacionales, como objeto-relacionales, etc�
\end{milista}

\subsection{Diferencias}
\begin{milista}
\item Web
	\begin{milista}
		\item Las bases de datos web deben disponer obligatoriamente de sistemas de seguridad.
		\item Deben estar orientadas a sesi�n.
		\item Las bases de datos web pueden estar basadas en XML, y para ello utilizan sistemas gestores de bases de datos nativos de XML (NXD, que utiliza el lenguaje XQuery).
	\end{milista}

\item Multimedia:
	\begin{milista}
		\item Mientras que en las bases de datos web debemos conocer exactamente como es el objeto que estamos intentando recuperar, en las multimedia no ocurre lo mismo, ya que las b�squedas pueden estar basadas en ejemplo y en las web recuperamos un objeto en concreto (conociendo de antemano sus caracter�sticas).
		\item De lo anterior puede derivarse que el uso de bases de datos multimedia aporta m�s funcionalidad al poder buscar por descriptores, similitud, metadatos, etc.
	\end{milista}
\end{milista}